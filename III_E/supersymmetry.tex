\documentclass[a4paper]{article}

\def\npart {III}
\def\nterm {Easter}
\def\nyear {2018}
\def\nlecturer {F.\ Quevedo}
\def\ncourse {Supersymmetry}

\usepackage{myheader}

\begin{document}
\maketitle
{\small
\setlength{\parindent}{0em}
\setlength{\parskip}{1em}
This course provides an introduction to the use of supersymmetry in quantum field theory. Supersymmetry combines commuting and anti-commuting dynamical variables and relates fermions and bosons.

Firstly, a physical motivation for supersymmetry is provided. The supersymmetry algebra and representations are then introduced, followed by superfields and superspace. 4-dimensional supersymmetric Lagrangians are then discussed, along with the basics of supersymmetry breaking. The minimal supersymmetric standard model will be introduced. If time allows a short discussion of supersymmetry in higher dimensions will be briefly discussed.

Three examples sheets and examples classes will complement the course.
\subsubsection*{Pre-requisites}
It is necessary to have attended the Quantum Field Theory and the Symmetries in Particle Physics courses, or be familiar with the material covered in them.
}
\tableofcontents

\section{Introduction}
What do we know so far? There are two basic theories in high energy physics --- special relativity and quantum mechanics. We know that quantum field theory gives us a consistent way to encapsulate both of these basic theories. In quantum field theory, particles are represented as excitations of a field $\phi(x)$. Particles come in two types, namely bosons and fermions. These have integer and half-integer spins respectively.

The basic and central example of a quantum field theory is the standard model. This has the following particles:
\begin{center}
  \begin{tabular}{cc}
    \toprule
    Particle & Spin\\
    \midrule
    Higgs & 0\\
    Quarks and leptons & $1/2$\\
    Gauge (photos, gluons, $W^{\pm}$, $Z$) & 1\\
    Nothing & $3/2$\\
    Gravitons & 2\\
    \bottomrule
  \end{tabular}
\end{center}
There are good reasons to believe that there aren't particles of spin greater than $2$. But we are still missing particles of spin $3/2$ in the standard model, which is curious.

In physics in general, a basic tool to understand our theory is symmetry. There are several types of symmetries:
\begin{itemize}
  \item Spacetime symmetries: $X^\mu \mapsto X'^\mu = \Lambda^\mu\!_\nu X^\nu + a^\mu$.
  \item Internal symmetries: $\phi \mapsto \phi' = \Omega \cdot \phi$ for some matrix $\Omega$. If $\Omega$ is constant, then this is said to be global. If they are spacetime dependent, they are said to be local.
\end{itemize}
Symmetries are useful for labeling and classifying particles, by mass, spin, charge, etc. Supersymmetry is a symmetry that relates fermions to bosons, and vice versa.

Symmetries are not here for fun. Gauge symmetries determine interactions, which are important in the Standard Model.

Unfortunately, symmetries are not always visible, but can hide from us, via spontaneous supersymmetry breaking. Thus, it is possible that there are a lot of symmetries around that we do not see. In fact, we have not seen any supersymmetry so far, but it is still fun to think about.
\subsection{History of SUSY}
In the 1960's, people discovered many hadrons. They could be organized in multiplets in the eightfold way.

In 1967, Coleman--Mandula provide that the symmetries of the $S$-matrix must be the direct product of the Poincar\'e group and internal symmetries, and they cannot be mixed. So you can't have symmetries that mix bosons and fermions because they act differently under the Poincar\'e group.

In 1971, Gelfand--Likhtman extended the Poincar\'e algebra, generated by $M_{\mu \nu}$ and $P_\mu$, to include include spinor generators, and this gives supersymmetry. This is something Coleman--Mandula missed. They assumed all the generators are not spinors.

At the same time, Ramond and Neveu--Schwarz, produced supersymmetry on the worldsheet of string theory, incorporating fermions in the theory.

In 1973, Volkov--Akulov thought it could be that neutrinos are Goldstone particles of some broken symmetry, which is supersymmetry.

In 1974, Wess--Zumino wrote down a supersymmetric field theory in four dimensions. This gives a well-defined subject for people to study. Then Salam--Strathdee came up with the notions of superfields and superspace.

In 1975, Haag--Lopuszanski--Sohnius generalized Coleman--Mandula to supersymmetry.

So far, this has been done without gravity.

In 1976, Freidman--van Nieuwenhuizen--Fesrora and Deser--Zumino came up with supergravity, with $s = \frac{3}{2}$ partners of gravitons.

Between 1977 and 1980s, people started to study supersymmetry phenomenology. For example, this gives a solution to the hierarchy problem, since supersymmetry protects certain quantities from quantum corrections.

From 1981 to 1984, Green--Schwarz developed superstring theory.

% 1991, data showed with SUSY, there is unification at high energies.

In 1994, Seiberg--Witten showed we can do non-perturbative $N = 2$ supersymmetry.

% 1996, black hole counting

In 1998, the AdS/CFT correspondence was discovered, and the conformal field theory in CFT is a supersymmetric field theory.
\section{SUSY Algebra and Representations}
\subsection{Poincar\'e symmetry and spinors} % fix this for html
In special relativity, the symmetries are given by the Poincar\'e group,
\[
  X^\mu \mapsto X'^\mu = \Lambda^\mu\!_\nu X^\nu + a^\mu,
\]
where $\Lambda^\mu\!_\nu$ is a Lorentz boost and $a^\mu$ is a translation. There is an invariant metric
\[
  \eta_{\mu\nu} = \eta^{\mu\nu} = \diag(+1, -1, -1, -1),
\]
with an invariant metric
\[
  \d s^2 = \eta_{\mu\nu}\;\d x^\mu\;\d x^\nu.
\]
Coordinate invariance means if $\Lambda$ is a Lorentz transform, then
\[
  \Lambda^T \eta \Lambda = \eta.
\]
This implies that $\det \Lambda = \pm 1$, and we also see that % how
\[
  (\Lambda^0\!_0)^2 - (\Lambda^1\!_0)^2 - (\Lambda^2\!_0)^2 - (\Lambda^3\!_0)^2 = 1.
\]
So we see that either $\Lambda^0\!_0 \geq 1$ or $\Lambda^0\!_0 \leq -1$.

In fact, the Lorentz group has four disconnected components, corresponding to the signs of $\det \Lambda$ and $\Lambda^0\!_0$. % Lorentz group is $\Or(3, 1)$.

\begin{defi}[Proper orthochronus group]\index{proper orthochronus group}\index{$\SO(3, 1)^\uparrow$}
  The \emph{proper orthochronus group} $\SO(3, 1)^\uparrow$ is the subgroup of the Lorentz group consisting of matrices $\Lambda$ with $\det \Lambda = 1$ and $\Lambda^0\!_0 \geq 1$.
\end{defi}
We have $\Or(3, 1) / \SO(3, 1)^{\uparrow} \cong \{1, [\Lambda_P], [\Lambda_T], [\Lambda_{PT}] \}$, where
\[
  \Lambda_P=
  \begin{pmatrix}
    +1 \\
    & -1\\
    & & -1\\
    & & & -1
  \end{pmatrix},\quad \Lambda_T =
  \begin{pmatrix}
    -1 \\
    & +1\\
    & & +1\\
    & & & +1
  \end{pmatrix},\quad
  \Lambda_{PT} = \Lambda_P \Lambda_T = -I.
\]
Infinitesimally, we can write
\[
  \Lambda^\mu\!_\nu = \delta^\mu\!_\nu + \omega^\mu\!_\nu, \quad a^\mu = \varepsilon^\mu.
\]
The condition that $\Lambda^T \mu \Lambda = \eta$ implies $\omega_{\mu\nu} = -\omega_{\nu\mu}$. So in total, we have $6 + 4$ parameters. So the Poincar\'e group has ten dimensions.

\subsubsection*{Poinacr\'e algebra}
What does the Poincar\'e algebra look like? We can describe it explicitly with generators and relations. If the Poincar\'e group acts on a Hilbert space via a representation $U = U(\Lambda, a)$, then we want to Taylor expand
\[
  U(1 + \varepsilon, \varepsilon) = 1 - \frac{i}{2} \omega_{\mu\nu} M^{\mu\nu} + i \varepsilon_\mu P^\mu.
\]
The point of the $i$'s is that if $U$ is unitary, then $M^{\mu\nu}$ and $P^\mu$ are Hermitian. These generate the Poincar\'e algebra.

Since translations commute, we have
\[
  [P_\mu, P_\nu] = 0.
\]
We next want to understand $[P^\sigma, M^{\mu\nu}]$. Since $P$ is a vector, under a Lorentz transformation, it transforms as
\[
  P^\sigma \mapsto \Lambda^\sigma\!_\rho P^\rho = (\delta^\sigma\!_\rho + \omega^\sigma_p) P^\rho = P^\sigma + \frac{1}{2} \omega_{\alpha \rho} (\eta^{\sigma \alpha} P^\rho - \eta^{\sigma \rho} P^\alpha).
\]
We can also think of $P^\sigma$ as an operator. Then
\begin{align*}
  P^\sigma \mapsto U^\dagger P^\sigma U &= \left(1 + \frac{i}{2} \omega_{\mu\nu} M^{\mu\nu}\right) P^\sigma \left(1 - \frac{i}{2} \omega_{\mu\nu} M^{\mu\nu}\right)\\
  &= P^\sigma - \frac{i}{2} \omega_{\mu\nu} [P^\sigma M^{\mu\nu} - M^{\mu\nu} P^\sigma].
\end{align*}
So we deduce that
\[
  [P^\sigma, M^{\mu\nu}] = i (P^\mu \eta^{\nu \sigma} - P^\nu \eta^{\mu \sigma}).
\]
Similarly, one calculates that
\[
  [M^{\mu\nu}, M^{\rho \sigma}] = i (M^{\mu\sigma} \eta^{\nu \rho} + M^{\nu \rho} \eta^{\mu\sigma} - M^{\mu \rho} \eta^{\nu \sigma} - M^{\nu \sigma} \eta^{\mu \rho}).
\]
\subsubsection*{Properties of the Lorentz group}
There is an isomorphism
\[
  \so(3, 1) \cong \su(2) \oplus \su(2).
\]
To see this, we define
\[
  J_i = \frac{1}{2} \varepsilon_{ijk} M_{jk},\quad K_i = M_{0i}.
\]
Observe that $J_i$ commute with $P^0$, the Hamiltonian, but $K_i$ do not. So $J_i$ are conserved quantities.

These then satisfy the commutation relations
\[
  [J_i, J_j] = i\varepsilon_{ijk} J_k,\quad [J_i, K_j] = i \varepsilon_{ijk} K_k,\quad [K_i, K_j] = -i \varepsilon_{ijk} J_k.
\]
The first commutation relation is familiar, coming from the rotation group, but the latter two are less familiar. If we further define
\[
  A_i = \frac{1}{2} (J_i + i K_i),\quad B_i = \frac{1}{2} (J_i - i K_i),
\]
then we get
\[
  [A_i, A_j] = i \varepsilon_{ijk} A_i,\quad [B_i, B_j] = \varepsilon_{ijk} B_k, [A_i, B_j] = 0.
\]
So in fact this Lie algebra is that of $\su(2) \oplus \su(2)$. Note that even though $J_i, K_i$ are Hermitian, $A_i$ and $B_i$ are not.

Using this isomorphism, we can classify all representations of $\so(3, 1)$. We write the labels as $(j_A, j_B)$, e.g.\ $(\frac{1}{2}, 0)$ means you tensor the smallest irrep of $A$ with the trivial representation of $B$, and similarly for $(0, \frac{1}{2})$. Moth of these have $j = j_A + j_B = \frac{1}{2}$.

$\SO(3, 1)$ is not simply connected. Instead, $\SL(2, \C)$ is a double (universal) cover of $\SO(3, 1)$. To see this, consider a vector $X = x_\mu e^\mu = (x_0, x_1, x_2, x_3)$. We can then define a corresponding element
\[
  \tilde{X} = X_\mu \sigma^\mu =
  \begin{pmatrix}
    x_0 + x_3 & x_1 - i x_2\\
    x_1 + i x_2 & x_0 - x_3,
  \end{pmatrix}
\]
where $\sigma^0 = I$ and $\sigma^i$ are the Pauli matrices. Explicitly,
\[
  \sigma^\mu = \left\{
    \begin{pmatrix}
      1 & 0\\
      0 & 1
    \end{pmatrix},
    \begin{pmatrix}
      0 & 1\\
      1 & 0
    \end{pmatrix},
    \begin{pmatrix}
      0 & -i\\
      i & 0
    \end{pmatrix},
    \begin{pmatrix}
      1 & 0\\
      0 & -1
    \end{pmatrix}
  \right\}.
\]
Since the $\sigma^\mu$ are linearly independent, $\tilde{X}$ and $X$ determine each other.

Recall the Lorentz group is defined by the property that the matrices preserve the metric $|X|^2 = x_0^2 - x_1^2 - x_2^2 - x_3^2$. This quantity is also the determinant of $\tilde{X}$. Since the action of $\SL(2, \C)$ on $\tilde{X}$ by conjugation $\tilde{X} \mapsto N \tilde{X} N^*$ preserves the determinant, by definition, there is a natural map $\SL(2, \C) \to \SO(3, 1)$ (since $\SL(2, \C)$ is connected), which we can explicitly check has kernel $\pm I$. So this is a double cover. % N^* is the componentwise complex conjugate

We can also explicitly check that
\[
  \begin{pmatrix}
    e^{i\theta/2} & 0\\
    0 & e^{-i\theta/2}
  \end{pmatrix}
\]
is rotation of $\theta$ around $x_3$.

If $N \in \SL(2, \C)$, then we can perform a \emph{polar decomposition} $N = e^H U$, where $H$ is Hermitian and $U$ is unitary. Since $H$ has three arbitrary parameters and $U$ takes values in $\SU(2) \cong S^3$, we know $\SL(2, \C) \cong \R^3 \times S^3$, and it is simply connected. % polar decomposition is unique?

\subsection{Representations and invariant tensors of \tph{$\SL(2, \C)$}{SL(2, C)}{SL(2, C)}} % proper tph
We now consider some representations of $\SL(2, \C)$.
\begin{defi}[Fundamental representation]\index{fundamental representation}
  The \emph{fundamental representation} of $\SL(2, \C)$ is the standard action of $\SL(2, \C)$ on $\C^2$. We write the elements as $\psi_\alpha$, which transform as
  \[
    \psi_\alpha \mapsto N_\alpha\!^\beta \psi_\beta
  \]
  for $\alpha, \beta = 1, 2$ and $N_\alpha\!^\beta \in \SL(2, \C)$. We say $\psi_\alpha$ is a \term{left-handed Weyl spinor}.
\end{defi}

\begin{defi}[Conjugate representation]\index{conjugate representation}
  The \emph{conjugate representation} is given by the action of $\SL(2, \C)$ on $\C^2$ given by
  \[
    \psi \mapsto \bar{N} \psi,
  \]
  where $\bar{N}$ is the element-wise complex conjugate of $N$. We write this in indices as
  \[
    \bar{\chi}_{\dot{\alpha}} \mapsto N^*_{\dot{\alpha}}\!^{\dot{\beta}} \bar{\chi}_{\dot{\beta}}.
  \]
  This is a \term{right-handed Weyl spinor}.
\end{defi}

\begin{defi}[Contravariant representations]\index{contravariant representations}
  We have
  \[
    \psi'^\alpha \mapsto \psi'^\beta (N^{-1})_\beta\!^\alpha,\quad \bar{\chi}'^{\dot{\alpha}} = \bar{\chi}^{\dot{\beta}} ((N^*)^{-1})_{\dot{\beta}}\!^{\dot{\alpha}}.
  \]
\end{defi}

For $\SO(3, 1)$, we used $\eta^{\mu\nu} = (\eta_{\mu\nu})^{-1}$ to raise and lower indices. For $\SL(2, \C)$, we use
\[
  \varepsilon^{\alpha\beta} = \varepsilon^{\dot{\alpha}\dot{\beta}} =
  \begin{pmatrix}
    0 & 1\\
    -1 & 0
  \end{pmatrix} = - \varepsilon_{\alpha\beta} = - \varepsilon_{\dot{\alpha}\dot{\beta}}.
\]
Essentially by definition of the determinant, $\varepsilon$ is invariant under $\SL(2, \C)$. This gives isomorphisms between the contravariant representations and the fundamental representations.

The map $\SL(2, \C) \to \SO(3, 1)$ tells us how we can treat $\SO(3, 1)$ representations as $\SL(2, \C)$ representations. Explicitly, given some $x_\mu$, we obtain $(x_\mu \sigma^\mu)_{\alpha \dot{\alpha}}$, where $\alpha$ and $\dot{\alpha}$ are unrelated indices. These transform as
\[
  (x_\mu \sigma^\mu)_{\alpha \dot{\alpha}} \mapsto N_\alpha\!^\beta (x_\nu \sigma^\nu)_{\beta \dot{\gamma}} (N^*)_{\dot{\alpha}}\!^{\dot{\gamma}}.
\]
We can also define
\[
  (\bar{\sigma}^\mu)^{\dot{\alpha} \alpha} \equiv = \varepsilon^{\alpha \beta} \varepsilon^{\dot{\alpha} \dot{\beta}} (\sigma^\mu)_{\beta \dot{\beta}} = (\mathbf{1}, -\boldsymbol\sigma).
\]
Note that the bar on $\bar{\sigma}^\mu$ has got nothing to do with complex conjugation!
\begin{ex}\leavevmode
  \begin{align*}
    \Tr(\sigma^\mu \bar{\sigma}^\nu) &= 2\eta^{\mu\nu}\\
    \sigma^\mu\!_{\alpha \dot{\alpha}} (\bar{\sigma}_\mu)^{\dot{\beta} \beta} &= 2 \delta_{\alpha}^\beta \delta_{\dot{\alpha}}^{\dot{\beta}}\\
    \sigma^\mu \bar{\sigma}^\nu + \sigma^\nu \bar{\sigma}^\mu &= 2\mathbf{1} \eta^{\mu\nu}.
  \end{align*}
\end{ex}

\subsection{Generators of \tph{$\SL(2, \C)$}{SL(2, C)}{SL(2, C)}} % fix tph
Infinitesimally, the spinors transform as
\begin{align*}
  \psi_\alpha &\mapsto (e^{-\frac{i}{2} (\omega_{\mu\nu} \sigma^{\mu\nu})})_\alpha\!^\beta \psi_\beta.\\
  \bar{\chi}^{\dot{\alpha}}&\mapsto (e^{-\frac{i}{2} \omega_{\mu\nu} \bar{\sigma}^{\mu\nu}})^{\dot{\alpha}}\!_{\dot{\beta}} \bar{\chi}^{\dot{\beta}},
\end{align*}
where
\begin{align*}
(\sigma^{\mu\nu})_\alpha\!^\beta &= \frac{i}{4} (\sigma^\mu \bar{\sigma}^\nu - \sigma^\nu \bar{\sigma}^\mu)_\alpha\!^\beta\\&=
  (\bar{\sigma}^{\mu\nu})^{\dot{\alpha}}\!_{\dot{\beta}} &= \frac{i}{4} (\bar{\sigma}^\mu \sigma^\nu - \bar{\sigma}^\nu \sigma^\mu)^{\dot{\alpha}}\!_{\dot{\beta}}.
\end{align*}
Thus, we have commutation relations
\[
  [\sigma^{\mu\nu}, \sigma^{\lambda\rho}] = i (\eta^{\mu\rho} \sigma^{\nu\lambda} + \eta^{\nu\lambda} \sigma^{\mu\rho} - \eta^{\mu\lambda} \sigma^{\nu\rho} - \eta^{\nu\rho} \sigma^{\mu\lambda}).
\]
Recall that we defined
\[
  J_i = \frac{1}{2} \varepsilon_{ijk} M_{jk}, \quad K_i = M_{0i}.
\]
Under the spinor representation, with $M^{\mu\nu} = \sigma^{\mu\nu}$, we get
\[
  J_i = \frac{1}{2} \sigma_i,\quad K_i = - \frac{i}{2} \sigma_i.
\]
So $A_i = \sigma_i,\quad B_i = 0$. So the fundamental representation is a $(\frac{1}{2}, 0)$ representation. Similarly, we see that under the conjugate representation, $A_i = 0,\quad B_i = \sigma_i$. So the conjugate representation is $(0, \frac{1}{2})$.

It will be convenient to note that
\begin{align*}
  \sigma^{\mu\nu} &= \frac{1}{2i} \varepsilon^{\mu\nu\rho\sigma} \sigma_{\rho\sigma}\\
  \bar{\sigma}^{\mu\nu} &= -\frac{1}{2i} \varepsilon^{\mu\nu\rho\sigma} \bar{\sigma}_{\rho\sigma}.
\end{align*}
We say $\sigma^{\mu\nu}$ is self-dual and $\bar{\sigma}^{\mu\nu}$ is anti-self dual. These imply that $\{\sigma^{\mu\nu}\}$ and $\{\bar{\sigma}^{\mu\nu}\}$ each only span a $3$-dimensional algebra.

\begin{notation}
  We write
  \[
    \chi\psi \equiv \chi^\alpha \psi_\alpha = - \chi_\alpha \psi^\alpha.
  \]
  Similarly,
  \[
    \bar{\chi} \bar{\psi} = \bar{\chi}_{\dot{\alpha}} \bar\psi^{\dot{\alpha}} = - \bar{\phi}^{\dot{\alpha}} \bar{\psi}_{\dot{\beta}}.
  \]
\end{notation}
In particular,
\[
  \psi\psi = \psi^\alpha \psi_\alpha = \varepsilon^{\alpha\beta} \psi_\alpha \psi_\alpha = \psi_2 \psi_1 - \psi_1 \psi_2.
\]
We choose $\psi_\alpha$ to be anti-commuting numbers. Then $\psi_1 \psi_2 = - \psi_2 \psi_1$, and so $\psi \psi = 2 \psi_2 \psi_1$. Then we have
\[
  \chi \psi = \chi^\alpha \psi_\alpha = - \chi_\alpha \psi^\alpha = + \psi^\alpha \chi_\alpha = \psi \chi.
\]
\begin{prop}[Fierz representations]\index{Fierz representation}
  \begin{align*}
    (\theta \psi) (\theta \psi) &= - \frac{1}{2} (\psi \psi) (\theta \theta) = - \frac{1}{2} (\theta \theta)(\psi \psi)\\
    \psi \sigma^{\mu\nu} \chi &= - \chi \sigma^{\mu\nu} \psi\\
    \psi_\alpha \bar{\chi}_\alpha &= \frac{1}{2} (\psi \sigma_\mu \bar{\chi}) \sigma^\mu\!_{\alpha \dot{\alpha}}\\
    \psi_\alpha \psi_\beta &= \frac{1}{2} \varepsilon_{\alpha\beta} (\psi \chi) + \frac{1}{2} (\sigma^{\mu\nu} \varepsilon^T)_{\alpha\betab} (\psi \sigma_{\mu\nu} \chi).
  \end{align*}
\end{prop}

What is the connection to Dirac spinors? We can define
\[
  \gamma^\mu =
  \begin{pmatrix}
    0 & \sigma^\mu\\
    \bar{\sigma}^\mu & 0
  \end{pmatrix},
\]
where each entry is a $2 \times 2$ matrix. So each $\gamma^\mu$ is a $4 \times 4$ matrix. Then $\sigma^\mu \bar{\sigma}^\nu + \sigma^\nu \bar{\sigma}^\mu = 2\mathbf{1}\eta^{\mu\nu}$ tells us
\[
  \{\gamma^\mu, \gamma^\nu\} = i \eta^{\mu\nu}.
\]
So $\{\gamma^\mu\}$ form a Clifford algebra representation.

We can also define
\[
  \gamma^5 = -i \gamma^0 \gamma^1 \gamma^2 \gamma^3 =
  \begin{pmatrix}
    -\mathbf{1} & 0\\
    0 & \mathbf{1}
  \end{pmatrix}.
\]
The eigenvalues of $\pm 1$ are the chirality.

\emph{Dirac spinors}\inde{Dirac spinor} are things of the form
\[
  \Psi_D = \begin{pmatrix}\psi_\alpha\\\bar{\chi}^{\dot{\alpha}}\end{pmatrix}.
\]
Then we have
\[
  \gamma_5 \Psi_D = \begin{pmatrix}\psi_\alpha\\\bar{\chi}^{\dot{\alpha}}\end{pmatrix}.
\]
There are projections $P_L = \frac{1}{2} (1 - \gamma_5)$ and $P_R = \frac{1}{2}(1 + \gamma_5)$, which give
\[
  P_L \Psi_D = \begin{pmatrix}\psi_\alpha\\ 0\end{pmatrix},\quad
  P_R \Psi_D = \begin{pmatrix}0\\\bar{\chi}^{\dot{\alpha}}\end{pmatrix}.
\]
Spinors of this form are also said to be \emph{Weyl}\index{Weyl spinor}.

Also, given a Dirac spinor, we can define the conjugate
\[
  \bar{\Psi}_D =
  \begin{pmatrix}
    \chi^\alpha & \bar{\psi}_{\dot{\alpha}}
  \end{pmatrix} = \psi_D^+ \gamma^0,\quad \Psi_D^c =
  \begin{pmatrix}
    \chi_\alpha \\ \bar{\psi}^{\dot{\alpha}}
  \end{pmatrix} = C \bar{\Psi}^T,
\]
where
\[
  C =
  \begin{pmatrix}
    \varepsilon_{\alpha\beta} & 0\\
    0 & \varepsilon^{\dot{\alpha} \dot{\beta}}
  \end{pmatrix}
\]
is the \term{charge conjugate} matrix.

A \term{Majorana spinor} is a Dirac spinor with $\psi_\alpha = \chi_\alpha$. So
\[
  \psi_M =
  \begin{pmatrix}
    \psi_\alpha\\ \bar{\psi}^{\dot{\alpha}}
  \end{pmatrix}.
\]
\subsection{The Supersymmetry Algebra}
A graded algebra is an algebra $A$ with a decomposition $A = A^0 \oplus A^1$. If $\mathcal{O}_a \in A^{\eta_a}$ etc., then we define % degree 0 = bosonic, degree 1 = fermionic.
\[
  [\mathcal{O}_a, \mathcal{O}_b] = \mathcal{O}_a \mathcal{O}_b - (-1)^{\eta_a \eta_b} \mathcal{O}_b \mathcal{O}_a. % we consider \{ \}
\]
If it has generators $\{\mathcal{O}_a\}$ each of a definite degree, then we write
\[
  [\mathcal{O}_a, \mathcal{O}_b] = i C^e_{ab} \mathcal{O}_e.
\]
In the supersymmetric extension of the Poincar\'e group, we have bosonic generators $P^\mu, M^{\mu\nu}$. We also introduce fermionic operators $Q_\alpha^A$ and $\bar{Q}_{\dot{\alpha}}^B$, where $\alpha, \dot{\alpha}$ are spinor indices, and $A, B = 1, 2, \ldots, \mathcal{N}$ are generic labels.

When $\mathcal{N} = 1$, we say we have \emph{simple supersymmetry}, and for $\mathcal{N} > 1$, it is extended supersymmetry.

We focus on the $\mathcal{N} = 1$. To determine the actual algebra structure, we need to know the commutators
\[
  [Q_\alpha, M^{\mu\nu}], [Q_\alpha, P^\mu], \{Q_\alpha, Q_\beta\}, [Q_\alpha, \bar{Q}_{\dot{\beta}}].
\]
If we have internal symmetries $T_i$, then we also want to know that $[Q_\alpha, T_i]$. Note that if we impose that $Q_\alpha$ and $\bar{Q}_{\dot{\alpha}}$ are ``conjugate'', then we get the remaining supersymmetry operators.

To determine the first, we use that $Q_\alpha$ is a spinor. So we want
\[
  Q_\alpha' = (e^{-\frac{i}{2} \omega_{\mu\nu} \sigma^{\mu\nu}})_\alpha^\beta Q_\beta = \left(1 - \frac{i}{2} \omega_{\mu\nu} \sigma^{\mu\nu}\right)_\alpha^\beta Q_B.
\]
As an operator, it transforms as
\[
  Q_\alpha' = U^\dagger Q_\alpha U = \left(1 + \frac{i}{2} \omega_{\mu\nu} M^{\mu\nu}) Q_\alpha \left(1 - \frac{i}{2} \omega_{\mu\nu} M^{\mu\nu}\right).
\]
Comparing both expressions, we get
\[
  [Q_\alpha, M^{\mu\nu}] = (\sigma^{\mu\nu})_\alpha\!^\beta Q_\beta.
\]
For the second, we see that the only possible combination is
\[
  [Q_\alpha, P^\mu] = c \sigma^\mu\!_{\alpha \dot{\alpha}} \bar{Q}^{\dot{\alpha}}.
\]
We have to determine $c$. Note that using $(\sigma \bar{Q})^\dagger = (Q\sigma)$, this implies
\[
  [\bar{Q}^{\dot{\alpha}}, P^\mu] = c^* (\bar{\sigma}^\mu)^{\dot{\alpha} \beta} Q_\beta.
\]
This does not follow immediately by ``complex conjugation'', because $\bar{\sigma}$ is not the complex conjugate of $\sigma$.

We now pull out the Jacobi identity
\[
  [A, [B, C]] + [C, [A, B]] + [B, [C, A]] = 0.
\]
with $A = P^\mu, B = P^\nu$ and $C = Q^_\alpha$. We then expand to get
\[
  |c|^2 (\sigma^\nu \bar{\sigma}^\mu - \sigma^\mu \bar{\sigma}^\nu) Q_\beta = 0.
\]
So we must have $c = 0$. So $[Q_\alpha, P^\mu] = 0$.

For the third, we again argue that we must have
\[
  \{Q_\alpha, Q^\beta\} = k (\sigma^{\mu\nu})_\alpha\!^\beta M_{\mu\nu}. % (1/2, 0) + (1/2, 0) = (0, 0) + (1, 0). No (0, 0) possibility.
\]
We have to find $k$. Observe that the left-hand side commutes with $P^\mu$. So the right-hand side must commute as well. So $k = 0$.

This is very particular to $\mathcal{N} = 1$. For larger $\mathcal{N}$, we can have scalar quantities.

For the fourth, we have
\[
  \{Q_\alpha, \bar{Q}_{\dot{\alpha}}\} = t \sigma^\mu_{\alpha \dot{\alpha}} P_\mu.
\]
It turns out there is no way to fix $t$, and the choice does not affect the physics (as long as it is non-zero, since we can always scale $Q$). The convention is that $t = 2$. So we get
\[
  \{Q_\alpha, \bar{Q}_{\dot{\alpha}}\} = 2 \sigma^\mu\!_{\alpha \dot{\alpha}} P_\mu.
\]
This says supersymmetry is a spacetime symmetry, because the action of two supersymmetric operators give us a translation.

Finally, internal symmetry generators, in general, commute with all spacetime symmetries, i.e.
\[
  [T_i, M^{\mu\nu}] = [T_i, P^\mu] = [T_i, Q_\alpha] = 0.
\]
There is an exceptions, known as \term{$R$-symmetry}. This is a $\U(1)$ automorphism of the supersymmetric algebra
\begin{align*}
  Q_\alpha &\mapsto e^{i\gamma} Q_\alpha\\
  \bar{Q}_{\dot\alpha} &\mapsto e^{-i\gamma} \bar{Q}_{\dot\alpha}.
\end{align*}
This has a $\U(1)$ generator $R$ with
\begin{align*}
  [Q_\alpha, R] &= Q_\alpha\\
  [Q_{\dot\alpha}, R] &= -\bar{Q}_{\dot\alpha}
\end{align*}
This says $Q_\alpha$ and $\bar{Q}_{\dot{\alpha}}$ have charge $+1$ and $-1$ respectively.

\subsection{Representations of the \tph{Poincar\'e}{Poincare}{Poincare} group} % fix tph
We first recall the representations of $\so(3)$. Recall that this has generators $J_i$ for $i = 1, 2, 3$, with commutation relation
\[
  [J_i, J_j] = i \varepsilon_{ijk} J_k.
\]
To understand the representations, it is convenient to consider the \term{Casimir operator}
\[
  J^2 = J_1^2 + J_2^2 + J_3^2.
\]
This operator satisfies
\[
  [J^2, J_i] = 0.
\]
We can pick our favorite value of $i$, say $i = 3$, and then label our states by the eigenvalues of $j$ and $j_3$. We write such a state as $|j, j_3\ket$, where $j_3 = -j, \ldots, j$.

In the Poincar\'e algebra, we have generators $M^{\mu\nu}$ and $P^\mu$. To proceed, we first find some Casimir operators, $C_1$ and $C_2$. We define them as
\[
  C_1 = P^\mu P_\mu,\quad C_2 = W^\mu W_\mu,\quad W^\mu = \frac{1}{2} \varepsilon_{\mu\nu\rho\sigma} P^\mu M^{\rho\sigma}.
\]
This mysterious object $W_\mu$ is called the \term{Pauli--Lubanski vector}. Note that this is not an element of the algebra itself. It is a straightforward (but tedious) exercise to check that these indeed commute with all the generators.

Each representation (multiplet) can thus be labelled by two numbers $m, w$, which are eigenvalues of $C_1, C_2$. Just as we had $j_3$ above, we would later seek more interesting labels.

One can check that
\[
  [W_\mu, P_\nu] = 0,\quad [W_\mu, W_\nu] = i \varepsilon_{\mu\nu\rho\sigma} W^\rho P^\sigma.
\]
As before, this is not a statement in the Lie algebra. Formally, we are working in the universal enveloping algebra.

To find more labels, we use that $[P^\mu, P^\nu] = 0$. So the $P^\mu$'s can be simultaneously diagonalized. So we have a new label $p^\mu$, and we can label our states as $|m, w, p^\mu\ket$.

We can find more generators. To make this easier to see, we pick a frame for $p^\mu$ and find generators that have it invariant. To be explicit, we change frame so that
\[
  p^\mu = (m, 0, 0, 0)
\]
if $C_1 = P^\mu P_\mu = 0$. Once we are in this frame, we see very clearly that $\so(3)$ leaves $p^\mu$ invariant. This $\so(3)$ is called the \term{little group} We can then label our state as
\[
  |m, j; p^\mu, j_3\ket. % j = w?
\]
If $C_1 = 0$, then we pick a frame where
\[
  p^\mu = (E, 0, 0, E).
\]
This corresponds to a massless particle. The little group is more complicated. Naively, we at least have an $\so(2)$ that acts in the middle. Observe that we have
\[
  (W_0, W_1, W_2, W_3) = E(J_3, -J_1 + K_2, -J_2 - K_1, -J_3).
\]
Then we find that
\[
  [W_1, W_2] = 0,\quad [W_3, W_1] = - i E W_2,\quad [W_3, W_2] = i E W_1.
\]
This algebra is called the \term{Euclidean group} in two dimensions. % this has infinite-dimensional representations (irreducible?)

When Wigner discovered this, he wasn't happy about the infinite-dimensional representations. If we require $W_1 = W_2 = 0$, then we are left with just the rotation part, and this has finite-dimensional representations. In this case, we find that $C_2 = 0$ too.

In this case, the only labels we are left with are $p_\mu$ and $\lambda$, the eigenvalue of $J_3$. The labels are $|0, 0; p_\mu, \lambda\ket = |p_\mu, \lambda\ket$.

We know that
\[
  e^{2\pi i \lambda} |p_\mu, \lambda\ket = \pm |p_\mu, \lambda\ket,
\]
we know that we have $\lambda \in \frac{1}{2} \Z$. % this is helicity?

\subsection{Representations of \texorpdfstring{$\mathcal{N} = 1$}{N = 1} SUSY}
In the $\mathcal{N} = 1$ SUSY algebra, we still have a Casimir
\[
  C_1 = P^\mu P_\mu.
\]
However, the old $C_2$ does not commute with the $Q_\alpha$. So it is no longer a Casimir. However, we can define
\begin{align*}
  B_\mu &= W_\mu - \frac{1}{4} \bar{Q}_{\dot{\alpha}} (\sigma_\alpha)^{\dot{\alpha} \beta } Q_\beta\\
  C_{\mu\nu} &= B_\mu P_\nu - B_\nu P_\mu\\
  \tilde{C}_2 = C_{\mu\nu} C^{\mu\nu}.
\end{align*}
Then $\tilde{C}_2$ is a Casimir operator.

\begin{prop}
  In any supersymmetric multiplet, the number of fermions is equal to the number of bosons. In short,
  \[
    n_F = n_B.
  \]
\end{prop}

\begin{proof}
  Consider the operator $(-1)^F$ defined by
  \[
    (-1)^F \bket{B} = \bket{B},\quad (-1)^F \bket{F} = -\bket{F}.
  \]
  Then we have
  \[
    \Tr (-1)^F = n_B - n_F.
  \]
  To compute the trace, we first observe that
  \[
    (-1)^F Q_\alpha \bket{F} = Q_\alpha \bket{F} = - Q_\alpha(-1)^F \bket{F},
  \]
  and similarly for bosons. So we find that
  \[
    \{(-1)^F, Q_\alpha\} = 0.
  \]
  Then in an irreducible representation, we have
  \begin{align*}
    \Tr \left[(-1)^F \{Q_\alpha, \bar{Q}_{\dot{\beta}}\}\right] &= \Tr \left[(-1)^F Q_\alpha \bar{Q}_{\dot{\beta}} + (-1)^F \bar{Q}_{\dot{\beta}} Q_\alpha\right]\\
    &= \Tr\{-Q_\alpha(-1)^F \bar{Q}_{\dot{\beta}} + Q_\alpha (-1)^F \bar{Q}_{\dot{\beta}}\}
    &= 0.
  \end{align*}
  On the other hand, we know
  \[
    \{Q_\alpha,\bar{Q}_{\dot{\beta}}\} = 2 \sigma^\mu_{\alpha \dot{\beta}} P_\mu.
  \]
  So we compute that
  \[
    \Tr \left[(-1)^F \{Q_\alpha, \bar{Q}_{\dot{\beta}}\}\right] = 2 \sigma^\mu\!_{\alpha\dot{\beta}} \Tr (-1)^F.
  \]
  So we must have $\Tr (-1)^F = 0$.
\end{proof}

\subsection{Massless supermultiplet}
Consider the case $p_\mu = (E, 0, 0, E)$. We then have $C_1 = \tilde{C}_2 = 0$. Restricting to such states, we have
\[
  \{Q_\alpha, \bar{Q}_{\dot{\beta}}\} = 2 \sigma^\mu_{\alpha \dot{\beta}} P_\mu = 2 E (\sigma^0 + \sigma^3) = 4E
  \begin{pmatrix}
    1 & 0\\
    0 & 0
  \end{pmatrix}_{\alpha \dot{\beta}}.
\]
This implies $Q_2 = 0$, and
\[
  \{Q_1, \bar{Q}_1\} = 4E.
\]
We now define
\[
  a = \frac{Q_1}{\sqrt{4E}},\quad a^\dagger = \frac{\bar{Q}_1}{\sqrt{4E}}.
\]
we then have
\[
  \{a, a^\dagger\} = 1,\quad \{a, a\} = \{a^\dagger, a^\dagger\} = 0.
\]
Also, we have
\[
  [a, J^2] = \frac{1}{2} \sigma_{11}^3 a = \frac{1}{2}a.
\]
So we find that
\[
  J^3a \bket{p^\mu, \lambda} = (aJ^3 - [a ,J])\bket{p^\mu, \lambda} = \left(aJ^3 - \frac{1}{2} a\right) \bket{p^\mu, \lambda} = \left(\lambda - \frac{1}{2}\right) a \bket{p^\mu \lambda}.
\]
So $a \bket{p^\mu, \lambda}$ is a state with helicity $\lambda - \frac{1}{2}$. Similarly, $a^\dagger \bket{p^\mu, \lambda} = \bket{p^\mu, \lambda - \frac{1}{2}}$.

Thus, to build a representation, start with a state $\bket{\Omega} = \bket{p^\mu, \lambda}$ such that $a \bke{\Omega} = 0$. The next state is then $a^\dagger \bket{\Omega} = \bket{p^\mu, \lambda + \frac{1}{2}}$. Since $(a^\dagger)^2 = 0$, we stop. % we try to build a finite-dimensional representation, or helicity is bounded below, this is dodgy

Thus, $\mathcal{N} = 1$ SUSY multiplets are two-dimensional, of the form % how about other momenta
\[
  \left\{\bket{p^\mu, \lambda}, \bket{p^\mu, ``l + \frac{2}{2}}\right\}.
\]
For $\lambda = 0$, we have the following pairings:
\begin{center}
  \begin{tabular}{cc}
    \toprule
    $\bket{p^{\mu}, 0}$ & $\bket{p^\mu, \frac{1}{2}}$\\
    \midrule
    Higgs & Higgsino\\
    squark & quark\\
    slepton & lepton\\
    \bottomrule
  \end{tabular}
\end{center}
These are known as chiral multiplets.

For $\lambda = \frac{1}{2}$, we have
\begin{center}
  \begin{tabular}{cc}
    \toprule
    $\bket{p^{\mu}, \pm \frac{1}{2}}$ & $\bket{p^\mu, \pm 1}$\\
    \midrule
    photino & photon\\
    wino & W\\
    zino & Z\\
    gluino & gluon\\ % capitalization
    \bottomrule
  \end{tabular}
\end{center}
These are known as gauge multiplets. Finally, for $\lambda = \frac{3}{2}$, we have
\begin{center}
  \begin{tabular}{cc}
    \toprule
    $\bket{p^{\mu}, \pm \frac{3}{2}}$ & $\bket{p^\mu, \pm 2}$\\
    \midrule
    gravitino & graviton\\
    \bottomrule
  \end{tabular}
\end{center}
This is the graviton multiplet.

\subsubsection*{Massive $\mathcal{N}=1$ multiplets}
Here we have
\[
  p^\mu = (m, 0, 0, 0),\quad C_1 = m^2,\quad \tilde{C}_2 = 2m^4 Y^i Y_i,
\]
where
\[
  Y_i = J_i - \frac{1}{4m} (\bar{Q} \sigma_i Q) = \frac{B_i}{m}.
\]
The eigenvalue of $Y$ is known as the \term{superspin}.

Observe that
\[
  [Y_i, Y_j] = i \varepsilon_{ijk} Y_k.
\]
The multiplet labels are $\bket{m, y}$, where $m$ comes from the eigenvalue of $C_1$ and $y$ is the eigenvalue of $Y^2$.

In this case, we have
\[
  \{Q_\alpha, \bar{Q}_{\dot{\beta}}\} = 2m \sigma^0_{\alpha \dot{\beta}} = 2m
  \begin{pmatrix}
    1 & 0\\
    0 & 1
  \end{pmatrix}_{\alpha \dot{\beta}}.
\]
This time $Q_1, Q_2$ are non-zero, and so we have two sets of creation and annihilation operators. We set
\[
  a_{1, 2} = \frac{Q_{1, 2}}{\sqrt{2m}}.
\]
Then we have
\[
  \{a_p, a_q^\dagger\} = \delta_{pq},\quad \{a_p, a_q\} = \{a_p^\dagger, a_q^\dagger\} = 0.
\]
If we start with a vacuum $\bket{\Omega}$, then
\[
  a_1 \bket{\Omega} = a_2 \bket{\Omega} = 0.
\]
In this case, we have
\[
  Y_i \bket{\Omega} = J_i \bket{\Omega}.
\]
So we have $y = j$. Then we have
\[
  \bket{\Omega} = \bket{m, j = y, p^\mu ,j_3}.
\]
For other states in the multiplet, $j$ will not be equal to $y$.

For example, for $y = 0$, we have
\begin{align*}a
  \Omega &= \bket{m, j = 0, p^\mu, j_3 = 0}\\
  a^\dagger_{1, 2} \bket{\Omega} &= \bket{m, j = \frac{1}{2}, p^\mu, j_3 = \pm \frac{1}{2}}\\
  a_1^\dagger a_2^\dagger \bket{\Omega} = \bket{m, j = 0, p^\mu, j_3 = 0}
\end{align*}
This forms an $\mathcal{N} = 1$, $y = 0$ multiplet. Also, observe that $\bket{\Omega'} = a_1^\dagger a_2^\dagger \bket{\Omega}$ has the same labels as $\bket{\Omega}$, but they are not the same element. Indeed, the latter is annihilated by $a_1$, but the former is that.

We also see that $\bket{\Omega}$ transforms in the $(1/2, 0)$ representation, while the latter transforms in the $(0, 1/2)$ representation. So they are related by parity. % why?

If $y \not= 0$, then we have a vacuum $\bket{\Omega}$ of spin $j$. Then $a_p^\dagger \bket{\Omega}$ transforms as $\frac{1}{2} \otimes j$, which decomposes as two irreducible representations.
\begin{align*}
  a_1^\dagger \bket{\Omega} &= k_1 \bket{m, j = y + 1/2, p^\mu, j_3 + 1/2} + k_2 \bket{m, j= y - 1/2, p^\mu, j_3 + 1/2}\\
  a_2^\dagger \bket{\Omega} &= k_3 \bket{m, j = y + 1/2, p^\mu, j_3 - 1/2} + k_4 \bket{m, j = y-1/2, p^\mu, j_3 - 1/2}\\
  a_2^\dagger a_1^\dagger \bket{\Omega} = - a_1^\dagger a_2^\dagger \bket{\Omega} = \bket{\Omega'}.
\end{align*}
At the end, we have two $\bket{m, j = y, p^\mu, j_3}$ states and one each of $\bket{m = j = y \pm 1/2, p^\mu, j_3}$. There are
\[
  2(2y + 1) + (2y + 2) + (2y)
\]
many states. So we see that
\[
  n_B = n_F = 4y + 2.
\]
\subsection{Extended supersymmetry}
We now consider supersymmetry with $\mathcal{N} > 1$. The supersymmetry operators satisfy (anti-)commutation relations
\begin{align*}
  \{Q_\alpha^A, Q_{\dot{\beta}, B}\} &= 2 \sigma^\mu_{\alpha\beta} P_\mu \delta^A\!_B\\
  \{Q_\alpha^A, Q_\beta\!^B\} = \varepsilon_{\alpha\beta} Z^{AB},
\end{align*}
where the $Z^{AB}$ are scalar operators called \emph{central charges}, which commute with $Q^A_\alpha, M^{\mu\nu}, P^\mu, Z^{CD}$, etc. Here $Z^{AB}$ has to be anti-symmetric, which is why it had to vanish for $\mathcal{N} = 1$. One check that this gives a genuine super Lie algebra.

Recall that we had $R$-symmetries. Here we have to be careful to make it preserve our second anti-commutation relations. If all $Z^{AB}$ are zero, then the inner automorphisms are of the form
\begin{align*}
  Q_\alpha^A &\mapsto U^A\!_B Q_\alpha^B\\
  Q_{\dot{\alpha}}^A &\mapsto (U^\dagger)^A\!_B \bar{Q}_{\dot{\alpha}}^B
\end{align*}
So this has a $U(\mathcal{N})$ symmetry. However, if $Z^{AB} \not= 0$, we have to inspect it manually.

If we have a massless representation, we get
\[
  \{Q_\alpha^A, \bar{Q}_{\dot{\beta}, B}\} = 4E
  \begin{pmatrix}
    1 & 0\\
    0 & 0
  \end{pmatrix}_{\alpha\beta}.
\]
So we must have $Q_2^A = 0$ and hence $Z^{AB} = 0$.

As before, we can define
\[
  a^A = \frac{Q_1^A}{2\sqrt{E}},\quad a^{A\dagger} = \frac{\bar{Q}_j}{2 \sqrt{E}},\quad\{a^A, a_B^T\} = \delta^A\!_B.
\]
We can tabulate the types and number of states:
\begin{tabular}{ccc}
  \toprule
  States & Helicity & Number of states\\
  \midrule
  $\bket{\Omega}$ & $\lambda_0$ & $1 = \binom{\mathcal{N}}{0}$\\
  $a^{A\dagger} \bket{\Omega}$ & $\lambda_0 + \frac{1}{2}$ & $\mathcal{N} = \binom{\mathcal{N}}{1}$\\
  $a^{A\dagger} a^{B\dagger} \bket{\Omega}$ & $\lambda_0 + 1$ & $\binom{\mathcal{N}}{2}$\\
  $\vdots$ & $\vdots$ & $\vdots$ \\
  $a^{\mathcal{N}\dagger} a^{(\mathcal{N} - 1)\dagger} \cdots a^{1\dagger} \bket{\Omega}$ & $\lambda_0 + \frac{\mathcal{N}}{2}$ & $1 = \binom{\mathcal{N}}{\mathcal{N}}$\\
  \bottomrule
\end{tabular}
We see that there are $2^{\mathcal{N}}$ states in total,

\begin{eg}
  If $\mathcal{N} = 2$, we have a \term{vector multiplet} with $\lambda_0 = 0$. This contains two $\mathcal{N} = 1$ chiral multiplets and two $\mathcal{N} = 1$ vector multiplets.
\end{eg}

\begin{eg}
  If $\mathcal{N} = 2$ again, we have a \term{hypermultiplet} with $\lambda_0 = -\frac{1}{2}$.
\end{eg}

In general, in every multiplet,
\[
  \lambda_{\mathrm{max}} - \lambda_{\mathrm{min}} = \frac{\mathcal{N}}{2}.
\]
However, in order for the theory to be renormalizable, we want $|\lambda| \leq 1$. So we often restrict to $\mathcal{N} \leq 4$.

Even if we don't care about renormalizability, then the maximum number of supersymmetries is $\mathcal{N} = 8$, since it would have more than one spin $2$ state (graviton), and there are also strong arguments against the existence of massless particles of helicities $> 2$. % soft theorems?

In general, for $\mathcal{N} > 1$ supersymmetry, we cannot avoid spin $1$ particles. So the theory cannot be chiral. The only exception is the hyper-multiplet, which doesn't have spin one particles, but we can manually see that is not chiral. But since real life is chiral, we stick to $\mathcal{N} = 1$.

%In real life, we see particles. Fields are introduced to describe local interactions among particles, say
%\[
%  \varphi(x), \psi_\alpha(x), A_\mu(x), \psi_\alpha^\mu (x) h_{\mu\nu}(x) % spin \lambda = 0, 1/2, 1, 3/2, 2.
%\]
%The relation between fields and particles is not one-to-one. For example, if we start with $A_\mu$, and write it as
%\[
%  A_\mu(x) = \int \frac{\d3 p}{\sqrt{E}} \varepsilon_\mu(p, \lambda) e^{ip\cdot x} a(p, \lambda) + \cdots,
%\]
%where $a(p, \lambda)$ is an annihilation operator, and $\varepsilon_\mu(p, \lambda)$ is a polarization.
%
%Observe that the particle $\bket{p^\mu, \pm 1}$ has two degrees of freedom, while the field $A_\mu$ has four degrees of freedom. What people do is that they impose constraints on the fields. We require
%\[
%  p^\mu \varepsilon_\mu = 0,
%\]
%which says the polarization is orthogonal to the momentum. We also have a redundancy that we can replace $\varepsilon_\mu$ with $\varepsilon_\mu + \alpha(p, \lambda) p_\mu$ for any $\alpha(p, \lambda)$.

\subsubsection{Massive representations}
For massive representations, if $Z^{AB} = 0$, then we get
\[
  \{Q_\alpha^A, \bar{Q}_{\beta, B}\} = 2m \delta^A\!_B
  \begin{pmatrix}
    1 & 0\\
    0 & 1
  \end{pmatrix}_{\alpha\beta}
\]
with $Q_1, Q_2 \not= 0$. Then there are $2^{2\mathcal{N}}$ states in a multiplet.

If $Z^{AB} \not= 0$, we can consider
\[
  \mathcal{H} \equiv (\bar{\sigma}^0)^{\dot{\beta \alpha}} \{Q_\alpha^A - \Gamma_\alpha\!^A \bar{Q}_{\dot{\beta} A} - \bar{\Gamma}_{\dot{\beta}, A}\} \geq 0,
\]
with
\[
  \Gamma_\alpha\!^A = \varepsilon_{\alpha\beta} U^{AB} \bar{Q}_{\dot{\gamma} B} (\bar{\sigma}^0)^{\dot{\gamma} \beta},
\]
where $U$ is a unitary matrix. From the supersymmetric algebra, we find that
\[
  \mathcal{H} = 8m\mathcal{N} - 2 \Tr(ZU^\dagger + U Z^\dagger).
\]
We now perform a polar decomposition of $Z$,
\[
  Z = HV,
\]
where $H$ is Hermitian and positive, and $V$ is unitary. We choose $U = V$. Then we get
\[
  \Tr(ZU^\dagger + UZ^\dagger) = \Tr (HVV^\dagger + UU^\dagger H) = 2 \Tr H.
\]
This implies
\[
  \mathcal{H} = 8m\mathcal{N} - 4 \Tr H \geq 0.
\]
Equivalently, we know that
\[
  m \geq \frac{\Tr \sqrt{Z^\dagger Z}}{2\mathcal{N}}.
\]
This gives a lower bound on the mass of particles. This is known as the \term{BPS bound}, by Bogomolny, Prasad and Sommerfeld.

If the bound is saturated, i.e.\ we have equality in the bound, then these are called \term{BPS states}. This is the case iff
\[
  Q_\alpha^A - \Gamma_\alpha^A = 0.
\]
This is important, the reason massless multiplets are smaller is that we had $Q_2 = 0$. Here a combination of $Q_i$ is smaller, and we again have smaller multiplets. Moreover, since BPS states are the lightest states of a given charge, they are stable.

For example, for $\mathcal{N} = 2$, we can take
\[
  Z^{AB} =
  \begin{pmatrix}
    0 & q_1\\
    -q_1 & 0
  \end{pmatrix}.
\]
We then get
\[
  m \geq \frac{q_1}{2}.
\]
\section{Superfields and \texorpdfstring{$\mathcal{N} = 2$}{N = 1} superspace}
Our goal is to write down a supersymmetric field theory. For $\mathcal{N} = 0$ we have fields of the form $\phi(x^\mu)$, where $x^\mu$ is a point in Minkowski space, and $\phi$ transforms under the Poincar\'e group.

For $\mathcal{N} = 1$ supersymmetry, we want a ``superfield'' $\Phi(X)$, where $X$ is a coordinate in superspace and $\Phi$ transforms under the super Poincar\'e algebra.

So what is superspace? Recall that a Lie group $G$ is in particular a manifold. To emphasize that we are thinking about it as a manifold, we write it as $M_G$.

\begin{eg}
  $\SL(2, \C) \cong \R^3 \times S^3$.
\end{eg}

If $H \leq G$ is a closed subgroup, then $G/H$ naturally has a manifold structure as well
\begin{eg}
  $\Or(n)/\Or(n - 1) \cong S^{n - 1}$ by the orbit stabilizer theorem.
\end{eg}

\begin{eg}
  $\U(1)$ embeds diagonally into $\SU(2)$, and $\SU(2)/\U(1) \cong S^2$, giving rise to the \term{Hopf fibration}.
\end{eg}

\begin{eg}
  Let $G$ be the Poincar\'e group, and $H$ be the Lorentz group. Then $G/H$ is Minkowski space, since $H$ is the stabilizer of the origin.
\end{eg}
We now have an $\mathcal{N} = 1$ super Poincar\'e ``group''. We can write an element as
\[
  g = 1 - i( \omega_{\mu\nu} M^{\mu\nu} - x_\mu P^\mu + \theta^\alpha Q_\alpha + \bar{\theta}_{\dot{\alpha}} \bar{Q}^{\dot{\alpha}}) + \cdots
\]
Quotienting this group by the Lorentz group then gives a manifold that is locally given by the parameters $\{x_\mu, \theta^\alpha, \bar{\theta}_{\dot{\alpha}}\}$. This is called $\mathcal{N} = 1$ superspace.

These $\theta^\alpha$ are anti-commuting numbers, and are known as \term{Grassmann variables}.

\subsection{Properties of Grassmann variables}
For simplicity, assume we have a single dimension $\theta$. Then an analytic function would have a Taylor expansion
\[
  f(\theta) = f_0 + f_1 \theta + f_2 \theta^2 + \cdots.
\]
However, since $\theta$ is anti-commuting, we know $\theta^2 = 0$. So the most general analytic function is just
\[
  f(\theta) = f_0 + f_1 \theta.
\]
This has ``derivative''\index{derivative}
\[
  \frac{\d f}{\d \theta} \equiv f_1.
\]
We require integrals to satisfy
\[
  \int \d \theta \left(\frac{\d f}{\d \theta}\right) = 0.
\]
which we can think of as Stokes' theorem. So we know that $\int \d \theta = 0$. To completely define the integral, we need to define what $\int \theta\;\d \theta$ is. We \emph{choose} this to be equal to one.

Thus, we see that we can define a ``Dirac delta function''
\[
  \delta(\theta) = \theta.
\]
Thus, we have
\[
  \int \d \theta\; f(\theta) = \int \d \theta\; (f_0 + f_1 \theta) = f_1 = \frac{\d f}{\d \theta}.
\]
If we have multiple Grassmann variables $\theta^\alpha$ and $\bar{\theta}_{\dot{\alpha}}$, we write
\[
  \theta \theta = \theta^\alpha \theta_\alpha,\quad \bar{\theta} \bar{\theta} = \bar{\theta}_{\dot{\alpha}} \bar{\theta}^{\dot{\alpha}}.
\]
We then have
\[
  \theta^\alpha \theta^\beta = -\frac{1}{2} \varepsilon^{\alpha\beta} (\theta \theta),\quad \bar{\theta}^{\dot{\theta}} \bar{\theta}^{\dot{\beta}} = \frac{1}{2} \varepsilon^{\dot{\alpha} \dot{\beta}} \bar{\theta} \bar{\theta}.
\]
We then set
\[
  \frac{\partial \theta^\beta}{\partial \theta^\alpha} = \delta_\alpha\!^\beta,\quad \frac{\partial\bar{\theta}^\dot{\beta}}{\partial \bar{\theta}^{\dot{\alpha}}} = \delta_{\dot{\alpha}}\!^\dot{\beta}.
\]
Then
\[
  \int \d \theta^1 \int \d \theta^2 \; \theta^2 \theta^1 = \frac{1}{2} \int \d \theta^1 \int \d \theta^2\; \theta\theta = 1.
\]
as a shorthand, we have
\[
  \int \d^2 \theta = \frac{1}{2} \int \d \theta^1 \int \d \theta^2.
\]
We can think of this as saying
\[
  \d^2 \theta = - \frac{1}{4} \d \theta^\alpha \d \theta^\beta \varepsilon_{\alpha\beta},\quad \d^2 \bar{\theta} = \frac{1}{4} \d \bar{\theta}^{\dot{\alpha}} \d \bar{\theta}^{\dot{\beta}} \varepsilon_{\dot{\alpha} \dot{\beta}}
\]
Then we have
\[
  \int \d^2 \theta = \frac{1}{4} \varepsilon^{\alpha\beta} \frac{\partial}{\partial \theta^\alpha} \frac{\partial}{\partial \theta^\beta},\quad \int^2 \bar{\theta} = - \frac{1}{4} \varepsilon^{\dot{\alpha}\dot{\beta}} \frac{\partial}{\partial \bar{\theta}^\dot{\alpha}} \frac{\partial}{\partial \bar{\theta}^{\dot{\beta}}}.
\]
\subsection{General scalar superfield}

Recall that for $\mathcal{N} = 0$, our scalar fields $\varphi(x^\mu)$ translate under translations by
\[
  \varphi \mapsto e^{-i a_\mu P^\mu} \varphi e^{ia_\mu P^\mu} = (1 - i a_\mu P^\mu) \varphi e^{i a_\mu P^\mu} = \varphi + i a_\mu [\varphi, P^\mu].
\]
On thee other hand, we have
\[
  \varphi \mapsto \vaprhi(x^\mu + a^\mu) = e^{ia_\mu \mathcal{P}^\mu} \varphi = \varphi(x) + \frac{\partial \varphi}{\partial x^``m} a_\mu.
\]
So we know that
\[
  \mathcal{P}^\mu = -i \frac{\partial}{\partial x^\mu}.
\]
Then we see that
\[
  \delta \varphi = i a_\mu [\varphi, P^\mu] i a_\mu \mathcal{P}^\mu \varphi = a_\mu \partial^\mu \varphi.
\]

For $\mathcal{N} = 1$, we consider a scalar field $S(x^\mu, \theta_\alpha, \bar{\theta}_{\dot{\alpha}})$. The most general expression for $S$ is
\begin{multline*}
  S(x^\mu, \theta_{\alpha}, \theta_{\dot{\alpha}}) = \varphi(x) + \theta \psi(x) + \bar{\theta} \bar{\chi}(x) + \theta \theta M(x) + \bar{\theta} \bar{\theta} N(x) + (\theta \sigma^\mu \bar{\theta}) V_\mu(x)\\
  + (\theta \theta) \bar{\theta} \bar{\lambda}(x) + (\bar{\theta} \bar{\theta}) \theta \rho(x) + (\theta \theta)(\bar{\theta} \bar{\theta}) D(x).
\end{multline*}
We see that this ``scalar field'' actually consists of many different fields $\varphi, \psi, \bar{\chi}, M, N, V_\mu, \bar{\lambda}, \rho, D$, some of which are bosonic and some of which are fermionic.

Under a supersymmetric transformation, we have
\[
  S \mapsto e^{-i(\varepsilon Q + \bar{\varepsilon} \bar{Q})} S(x, \theta, \bar{\theta}) e^{i (\varepsilon Q + \bar{\varepsilon}) \bar{Q}} = e^{i (\varepsilon \mathcal{Q} + \bar{\varepsilon} \bar{\mathcal{Q}})} S(x, \theta, \bar{\theta}).
\]
We then find that
\[
  \delta S = i [S, \varepsilon Q + \bar{\varepsilon} \bar{Q}] = i (\varepsilon \mathcal{Q} + \bar{\varepsilon} \bar{\mathcal{Q}}) S(x, \theta, \bar{\theta}).
\]
Our goal is then to find $Q_\alpha$ and $\bar{Q}_{\dot{\alpha}}$.

We assert that we have
\[
  S(x, \theta, \bar{q}) = e^{i(x^\mu P_\mu, \theta Q + \theta \bar{Q})} S(0, 0, 0) e^{i (x^\mu P_\mu + \theta Q + \bar{\theta} \bar{Q})}.
\]
Then the BCH formula
\[
  e^A e^B = e^{A + B + [A, B] + \cdots}
\]
tells us we get
\[
  S \mapsto S(x -i \varepsilon \sigma^\mu \bar{\theta} + i \theta \sigma^\mu \bar{\varepsilon}, \theta + \varepsilon, \theta + \bar{\varepsilon})
\]
Taylor expanding, we get
\begin{align*}
  \mathcal{Q}_\alpha &= -i \frac{\partial}{\partial \theta^\alpha} + (\sigma^\mu)_{\alpha \dot{\beta}} \bar{\theta}^{\dot{\beta}} \frac{\partial}{\partial x^\mu}\\
  \bar{\mathcal{Q}}_{\dot{\alpha}} &= i \frac{\partial}{\partial \bar{\theta}^{\dot{\alpha}}} + \theta^\beta (\alpha^\mu)_{\beta \dot{\alpha}} \frac{\partial}{\partial x^\mu}\\
  \mathcal{P}_\mu &= -i \partial_\mu
\end{align*}
One can then verify that we have
\begin{align*}
  \{\mathcal{Q}_\alpha, \bar{\mathcal{Q}}_{\dot{\alpha}} &= 2 (\sigma^\mu)_{\alpha \dot{\alpha}} \mathcal{\mathcal{P}}_\mu\\ % sign because \varepsilon Q = \varepsilon^\alpha Q_\alpha and \bar{\varepsilon}\bar{Q} = \bar{\varepsilon}_{\dot{\alpha}} \bar{Q}^{\dot\alpha}
  \{\mathcal{Q}_\alpha, \mathcal{Q}_\beta\} &= 0\\
  [\mathcal{Q}_\alpha, \mathcal{P}_\mu] = 0.
\end{align*}
We can now read off how each of the individual components transform:
\begin{align*}
  \delta \varphi &= \varepsilon \psi + \bar{\varepsilon} \bar{\chi}\\
  \delta \psi &= 2 \varepsilon M + \sigma^\mu \bar{\varepsilon} (i \partial_\mu \varphi + V_m)\\
  \delta \bar{\chi} &= 2 \bar{\varepsilon} N - \varepsilon \sigma^\mu (i \partial_\mu \varphi - V_\mu)\\
  \delta M &= \bar{\varepsilon} \bar{\lambda} - \frac{i}{2} \partial_\mu \psi \sigma^\mu \bar{\varepsilon}\\
  \delta N &= \varepsilon \rho + \frac{i}{2} \varepsilon \sigma^\mu \partial_\mu \bar{\chi}\\
  \delta V_\mu &= \varepsilon \sigma_\mu \bar{\lambda} + \rho \sigma_\mu \bar{\varepsilon} + \frac{i}{2} (\partial^\nu \psi \sigma_\mu \bar{\sigma}_\nu \varepsilon - \bar{\varepsilon} \bar{\sigma}_\nu \sigma_\mu \partial^\nu \bar{\chi})\\
  \delta \bar{\lambda} &= 2 \bar{\varepsilon} D + \frac{i}{2} (\bar{\sigma}^\nu \sigma^\mu \bar{\varepsilon})\partial_\mu V_\nu + i \bar{\sigma}^\mu \varepsilon \partial_\mu M\\
  \delta \rho &= 2 \varepsilon D - \frac{i}{2} (\bar{\sigma}^\nu \sigma^\mu \varepsilon) \partial_\mu V_\nu + i \bar{\sigma}^\mu \varepsilon \partial_\mu M\\
  \delta D &= \frac{i}{2} \partial_\mu (\varepsilon \sigma^\mu \bar{\lambda} - \rho \sigma^\mu \bar{\varepsilon}).
\end{align*}
Above all, note that the change in the last component $D$ is a total derivative!

Note that the product of any two superfields is a superfield, transforming infinitesimally under
\[
  \delta (S_1 S_2) = i [S_1 S_2, \varepsilon Q + \bar{\varepsilon} \bar{Q}] = i S_1 [S_2, \varepsilon Q + \bar{\varepsilon} \bar{Q}] + i [S_1, \varepsilon Q + \bar{\varepsilon} \bar{Q}] S_2.
\]
Note that $\partial_\mu S$ is a superfield, but $\partial_\alpha S$ is \emph{not} a superfield. This is since $\partial_\mu$ commutes with $Q_\alpha$ and $\bar{Q}_{\dot{\alpha}}$, but $\partial_\dot{\alpha}$ does not. Thankfully, we can define a covariant derivative
\[
  \mathcal{D}_\alpha = \partial_\alpha + i (\sigma^\mu)_{\dot{\alpha} \dot{\beta}} \bar{\theta}^{\dot{\beta}} \partial_\mu,\quad \bar{\mathcal{D}}_{\dot{\alpha}} = \bar{\partial}_{\dot{\alpha}} + i \theta^\beta (\sigma^\mu)_{\beta \dot{\alpha}} \partial_\mu.
\]
One then checks that
\[
  \{\mathcal{D}_\alpha, \mathcal{Q}_\beta\} = \{\mathcal{D}_\alpha \bar{\mathcal{Q}}_{\beta}\} = \{\bar{\mathcal{D}}_\alpha, \mathcal{Q}_\beta\} = \{\bar{\mathcal{D}}_\alpha, \bar{\mathcal{Q}}_{\dot{\beta}}\} = 0.
\]
Therefore
\[
  [\mathcal{D}_\alpha, \varepsilon \mathcal{Q} + \bar{\varepsilon} \bar{\mathcal{Q}}] = 0.
\]
So $\mathcal{D}_\alpha S$ and $\bar{\mathcal{D}}_{\dot{\alpha}} S$ are superfields.

Observe that if $S = f(x)$ is a superfield, then $f$ is constant.

Note that the number of degrees of freedom of $S$ is much greater than what we saw when we studied representation theory. This suggests $S$ does \emph{not} give an irreducible representation of the supersymmetric field. By imposing some conditions on $S$, we can get smaller representations.
\begin{itemize}
  \item \term{chiral superfields} $\Phi(x, \theta, \bar{\theta})$ are superfields that satisfy $\bar{\mathcal{D}}_{\dot{\alpha}} \Phi = 0$
  \item \term{anti-chiral superfields} $\Phi(x, \theta, \bar{\theta})$ satisfy $\mathcal{D}_\alpha \Phi = 0$.
  \item \term{vector superfields} $V$ satisfies $V = V^\dagger$
  \item \term{linear superfields} $L$ satisfy $LL^\dagger$ and $\mathcal{D} \mathcal{D} L = 0$.
\end{itemize}
We will see that chiral and anti-chiral superfields are ``matter'' fields; vector superfields are the gauge fields, and linear superfields are in some sense the same as chiral superfields under duality transformations.

\subsection{Chiral superfields}
To simplify the covariant derivative, we absorb some of it by defining.
\[
  y^\mu = x^\mu + i \theta \sigma^\mu \bar{\theta}.
\]
Then we can express the chirality condition as
\[
  0 = \bar{\mathcal{D}}_\alpha \Phi = \bar{\partial}_{\dot{\alpha}} \Phi + \frac{\partial \Phi}{\partial y^\mu} \frac{\partial y^\mu}{\partial \bar{\theta}^\mu} + i \theta^\beta \sigma^\mu_{\beta \dot{\alpha}} \partial_\mu \Phi = \bar{\partial}_{\dot{\alpha}} \Phi
\]
So $\Phi$ is chiral iff $\Phi = \Phi(y, \theta)$. Then we can write
\[
  \Phi(y, \theta) = \varphi(y^\mu) + \sqrt{2} \theta \psi(y^\mu) + \theta \theta F(y^\mu).
\]
This is far simpler than what we had before!

Observe that $\varphi$ and $\psi$ have four real degrees of freedom, and $\psi$ has four real degrees of freedom. In terms of $x$, this can be written as
\[
  \Phi(x, \theta, \bar{\theta}) = \varphi(x) + \sqrt{2} \theta \psi(x) + \theta \theta F(x) + i \theta \sigma^\mu \bar{\theta} \partial_\mu \varphi(x) - \frac{i}{\sqrt{2}} (\theta\theta)\partial_\mu \psi(x) \sigma^\mu \bar{\theta} - \frac{1}{4} (\theta\theta)(\bar{\theta} \bar{\theta}) \partial_\mu \partial^\mu \varphi(x).
\]
Then under a supersymmetric transformation, we have
\begin{align*}
  \delta \varphi &= \sqrt{2} \varepsilon \psi\\
  \delta \psi &= i \sqrt{2} \sigma^\mu \bar{\varepsilon} \partial_\mu \varphi + \sqrt{2} \varepsilon F\\
  \delta F &= i \sqrt{2} \bar{\varepsilon} \bar{\sigma}^\mu \partial_\mu \psi.
\end{align*}
Observe that similar to $D$, the quantity $F$ changes by a total derivative. We will see that $(\varphi, \psi)$ is a chiral $\mathcal{N} = 1$ multiplet, and $F$ is an auxiliary field.

Observe that the Leibniz rule says the product of chiral superfields is chiral. Hence any holomorphic function of chiral superfields is chiral. Also, if $\Phi$ is chiral, then $\Phi^\dagger$ is anti-chiral.

Also if $\Phi$ is chiral, then $\Phi^\dagger \Phi$ and $\Phi^\dagger + \Phi$ are real superfields.

Also, if $X = (x, \psi_x, F_x)$ is chiral and nilpotent, i.e.\ $X^2 = 0$, then
\[
  x^2 + 2 \sqrt{2} x \theta \psi_x + (2x F_x - \psi^2_x)\theta \theta = 0.
\]
This requires $x = \psi_x^2/2F_x$.

\subsection{Vector superfields}
We now consider vector superfields, satisfying
\[
  V(x, \theta, \bar{\theta}) = V^\dagger (x, \theta, \bar{\theta}).
\]
This can be expressed as
\begin{multline*}
  V = C(x) + i \theta \chi(x) - i \bar{\theta} \bar{\chi}(x) + \frac{i}{2} \theta \theta (M(x) + i N(x)) - \frac{i}{2} \bar{\theta} \theta (M - iN) + \theta \sigma^\mu \theta V_\mu +\\
  i \theta \theta \bar{\theta} \left(-i \bar{\lambda}(x) + \frac{i}{2} \bar{\sigma}^\mu \partial_\mu \chi(x)\right) - i \bar{\theta} \bar{\theta} \theta \left(i \lambda(x) - \frac{i}{2} \sigma^\mu \partial_\mu \bar{\chi}(x)\right) + \frac{1}{2} (\theta \theta)(\bar{\theta} \bar{\theta}) \left(d - \frac{1}{2} \partial_\mu \partial^\mu C\right).
\end{multline*}
One can check that there are 8 bosonic components, $C, M , N, D, V_\mu$, and eight fermionic components $\chi_\alpha, \lambda_\alpha$.

Notice that if $\Lambda$ is chiral, then the imaginary part $i(\Lambda - \Lambda^\dagger)$ is a vector superfield. We see that the components of $i(\Lambda - \Lambda^\dagger)$ are
\begin{align*}
  C &= i (\varphi - \varphi^\dagger)\\
  \chi &= \sqrt{2} \psi\\
  \frac{1}{2} (M + i N) &= F\\
  V_\mu &= - \partial_\mu (\varphi + \varphi^\dagger)\\
  \lambda = D = 0
\end{align*}
Note that $V_\mu$ is a total derivative of a real function. Define a generalized gauge transformation of a superfield $V$ by
\[
  V \mapsto V - \frac{i}{2} (\Lambda - \lambda^\dagger).
\]
The \term{Wess-Zumino gauge} chooses components of $\Lambda$ such that $C = M = N = 0$. This is clearly possible. Of course, we will have to use Lagrangians that are gauge invariant. Then
\[
  V_{WZ} (x, \theta, \bar{\theta}) = (\theta \sigma^\mu \bar{\theta}) V_\mu(x) + (\theta\theta)(\bar{\theta} \bar{\lambda}(x)) + \bar{\theta} \bar{\theta} (\theta\lambda(x)) + \frac{1}{2} (\theta\theta)(\bar{\theta}\bar{\theta}) D(x).
\]
So the components are $(V_\mu(x), \lambda(x), D(x))$. Here $V_\mu$ is the gauge field, $\lambda(x)$ is the gaugino and $D(x)$ is an auxiliary field.

Note that we have
\[
  V_{WZ}^2 = \frac{1}{2} \theta \theta \bar{\theta} \bar{\theta} V^\mu V_\mu,
\]
and higher powers of $V_{WZ}$ vanish.

Recall that for gauge fields, we had field strengths. We can do that here as well. To avoid complications involving the commutator, we shall only do the abelian case. We define
\[
  \mathcal{W}_\alpha = - \frac{1}{4} (\bar{\mathcal{D}} \bar{\mathcal{D}}) \mathcal{D}_\alpha V.
\]
Note that this is chiral and gauge invariant. In terms of $y$, we have
\[
  \mathcal{W}_\alpha(y, \theta) = \lambda_\alpha(y) + \theta_\alpha D(y) + (\sigma^{\mu\nu} \theta)_\alpha F_{\mu\nu}(y) - i (\theta \theta) \sigma^\mu_{\alpha \dot{\beta}} \partial_\mu \bar{\lambda}^\beta(y),
\]
where
\[
  F_{\mu\nu} = \partial_\mu V_\nu - \partial_\nu V_\mu.
\]
\section{Supersymmetric Lagrangian}
\subsection{\texorpdfstring{$\mathcal{N} = 1$}{N = 1} global SUSY}
Our goal is to write down a Lagrangian for interactions of chiral and vector superfields, $\mathcal{L}[\Phi, V, \mathcal{W}_\alpha]$. We first consider Lagrangians for chiral superfields.

Recall that for a general scalar superfield, the component $D$ transforms as a total derivative
\[
  \delta D = \frac{i}{2} \partial_\mu (\varepsilon \sigma^\mu \bar{\lambda} - \rho \sigma^\mu \varepsilon).
\]
For a chiral superfield, $F$ transforms as a total derivative
\[
  \delta F = i \sqrt{2} \bar{\varepsilon} \bar{\sigma}^\mu \partial_\mu \psi.
\]
The most general Lagrangian for $\Phi$ is
\[
  \mathcal{L} = K(\Phi, \Phi^\dagger)|_D + (W(\Phi)|_F + \text{h.c.})
\]
where the $|_D$ thing says we take the $D$ part, and same for $F$. Then under a supersymmetric transformation, $\mathcal{L}$ transforms by a total derivative. This $K$ is called the \term{K\"ahler potential}, and $W$ is the \term{superpotential}.

We look for renormalizable Lagrangians. For this, we need to understand the units of the superfields. We have
\[
  [\mathcal{L}] = 4, [\varphi] = 1, [\Phi] = 1, [\psi] = \frac{3}{2}, [\theta] = -\frac{1}{2}, [F] = [2].
\]
For $\mathcal{L} = 4$, we want
\[
  K = \cdots + K_D \theta \theta (\bar{\theta} \bar{\theta}),
\]
with $[K_D] \leq 4$. So we must have
\[
  [K] \leq 2.
\]
Similarly, we need that
\[
  [W] \leq 3.
\]
\begin{eg}
  We can take
  \[
    K = \Phi^\dagger \Phi,\quad W = \alpha + \lambda \Phi + \frac{m}{2} \Phi^2 + \frac{g}{3} \Phi^3.
  \]
  These choices define the \term{Wess--Zumino model}. Explicitly, % fill in
%  \[
%     \partial^\mu \varphi \partial_\mu \varphi^* + \bar\psi \sigma^\mu \partial_\mu \psi + F^2.
%  \]
\end{eg}
For $W(\Phi)$, it is convenient to expand around $\Phi = \varphi$.
\[
  W(\Phi) = W(\varphi) + (\Phi - \varphi) \frac{\partial W}{\partial \varphi} + \frac{1}{2} (\Phi - \varphi)^2 \frac{\partial^2 W}{\partial \varphi^2} + \cdots.
\]
One finds that the Lagrangian for $F$ is
\[
  \mathcal{L}[F] = FF^* + F^* \frac{\partial W}{\partial \varphi} + F^* \frac{\partial W^*}{\partial \varphi^*}.
\]
If we solve for $F$ by setting $\frac{\delta \mathcal{L}}{\delta F} = 0$, then we get
\begin{align*}
  F^* + \frac{\partial W}{\partial \varphi} &= 0\\
  F + \frac{\partial W^*}{\partial \varphi^*} &= 0.
\end{align*}
Plugging this back in, we get
\[
  \mathcal{L}[F] = - \left|\frac{\partial W}{\partial \varphi}\right|^2.
\]
This gives a scalar potential for $\varphi$,
\[
  V = \left|\frac{\partial W}{\partial \varphi}\right|^2 \geq 0.
\]
Note that
\begin{itemize}
  \item The $\mathcal{N} = 1$ Lagrangian is a particular case of a $\mathcal{N} = 0$ Lagrangian.
  \item The scalar potential for $\varphi$ is automatically always non-negative.
  \item The boson and fermion have the same mass.
  \item There is a quartic term in the potential $g^2 |\varphi|^4$.
  \item There is also a Yukawa coupling with the same coupling $g$, which allows for ``miraculous cancellation'' between $g^2 |\varphi|^4$ and two $g \psi^2 \varphi$ couplings. % insert diagram
\end{itemize}

In general, if we have several fields $\Phi_i$, the kinetic term would be
\[
  K_{i\bar{j}} \partial^\mu\varphi^i \partial_\mu \varphi^{\bar{j}},
\]
where
\[
  K_{i\bar{j}} = \frac{\partial^2 K}{\partial \varphi_i \partial \varphi_{\bar{j}}}.
\]
In this case, the potential is
\[
  V_{(F)} = K_{i\bar{j}}^{-1} \frac{\partial W}{\partial \varphi_i} \frac{\partial \bar{W}}{\partial \varphi_{\bar{j}}} \geq 0.
\]
\subsection{\texorpdfstring{$\mathcal{N} = 1$}{N = 1} super QED}
Recall that to do $\U(1)$ gauge theory, we start with a Lagrangian $\mathcal{L} = \bar{\psi} \slashed\partial \psi$, we promote the global $\U(1)$ symmetry to a local symmetry $\psi \mapsto e^{i \lambda} \psi$, where $\lambda = \lambda(x)$. Thus, our new Lagrangian is
\[
  \mathcal{L} = \bar{\psi} \slashed{D} \psi,\quad D_\mu = \partial_\mu - i q A_\mu.
\]
We then add a kinetic term $\frac{1}{4} F_{\mu\nu} F^{\mu\nu}$ for the potential, where $F_{\mu\nu} = \partial_\mu A_\nu - \partial_\nu A_\mu$.

For $\mathcal{N} = 1$ supersymmetry, we start with
\[
  \mathcal{L} = \Phi^\dagger \Phi|_D.
\]
We want symmetry under $\Phi \to e^{i q \Lambda} \Phi$, where $\Lambda$ is a chiral superfield. Then the Lagrangian transforms as
\[
  \mathcal{L} \mapsto \Phi^\dagger e^{-i q (\Lambda^\dagger - \Lambda)} \Phi|_D.
\]
To make it invariant, we introduce a real vector superfield $V$ that transforms as
\[
  V \mapsto V - \frac{i}{2} (\Lambda - \Lambda^\dagger),
\]
Then the new Lagrangian
\[
  \mathcal{L} = \Phi^\dagger e^{iqV} \Phi|_D
\]
is invariant. Note that in the Wess--Zumino gauge, we simply have
\[
  e^{2qV_{WZ}} = 1 + 2qV_{WZ} + 2q^2 V_{WZ}^2.
\]
In this gauge, $\Phi= \{ \varphi, \psi, F\}$ and $V = \{\lambda, V_\mu, D\}$, and
\[
  \mathcal{L} = F^* F + \D_\mu \varphi \D^\mu \varphi^* + i \bar{\psi} \bar{\sigma}^\mu \D_\mu \psi + \sqrt{2} q(\varphi \bar{\lambda} \bar{\psi} + \varphi^* \lambda \psi) + q D |\varphi|^2,
\]
where
\[
  \D_\mu = \partial_\mu - i q V_\mu.
\]
Note that the Wess--Zumino gauge is not preserved by supersymmetric transformations.

To introduce the kinetic part of the field strength, we use
\[
  \mathcal{W}^\alpha \mathcal{W}_\alpha|_F = D^2 - \frac{1}{2} F_{\mu\nu} F^{\mu\nu} - 2i \lambda \sigma^\mu \partial_\mu \bar{\lambda} - \frac{i}{2} F_{\mu\nu} \tilde{F}^{\mu\nu},
\]
where $\tilde{F}^{\mu\nu} \varepsilon^{\mu\nu\rho\sigma} F_{\varphi \sigma}$. Note that $F_{\mu\nu} \tilde{F}^{\mu\nu}$ is a total derivative term, so sometimes people don't talk about it, but it is important for, say, instantons. % this term vanishes when we take hermitian conjugate.

Then the total Lagrangian is
\[
  \mathcal{L}_{\mathrm{total}} = (\Phi^\dagger e^{2qV} \Phi)_D + ((\mathcal{W}^\alpha \mathcal{W}_\alpha)_F + \text{h.c.}).
\]
Are there more terms we can add to the Lagrangian? Fayet and Iliopoulos introduced the \term{Fayet--Iliopoulos term}
\[
  \mathcal{L}_{FI} = \xi V_D,
\]
where $\xi$ is a number. Note that this is gauge invariant only for an abelian theory. The dependence of $\mathcal{L}_{\mathrm{total}}$ on $D$ is now
\[
  \mathcal{L}[D] = D^2 + q |\varphi|^2 D + \frac{1}{2} \xi D.
\]
This has no kinetic term, and hence doesn't propagate. We can solve for $D$ to get
\[
  D = - \frac{\xi}{2} - q |\varphi|^2.
\]
Then we get
\[
  \mathcal{L}_[D] = - \frac{1}{8} (\xi + 2q|\varphi|^2)^2 \equiv -V_{(D)}.
\]
So the total potential is
\[
  V = V[F] + V[D] = \left|\frac{\partial W}{\partial \varphi}\right|^2 + \frac{1}{8} (\xi + 2q|\varphi|^2)^2 \geq 0.
\]

More generally, we can take
\[
  \mathcal{L} = K(\Phi^\dagger, e^{iqV}, \phi)|_D + (f(\Phi_i) \mathcal{W}^\alpha \mathcal{W}_\alpha + \text{h.c.}) + (W(\Phi_i)_F + \text{h.c.}) + \xi V_D.
\]
Here $f(\Phi)$ is a holomorphic function in $\Phi$, and the middle term is called the gauge kinetic function.

Given a Lagrangian $\mathcal{L}$, we can define the action as a superspace integral. For $\mathcal{N} = 0$, we have
\[
  S = \int \d^4 x\; \mathcal{L}.
\]
For $\mathcal{N} = 1$, recall that we had
\[
  \int \d^2 \theta\; \theta \theta = 1,\quad \int\d^4 \theta (\theta\theta)(\bar{\theta} \bar{\theta}) = 1.
\]
Then
\[
  S = \int \d^4 x\; \left[\int \d^4 \theta\; (K + \xi V) + \int \d^2 \theta (W(\Phi) + f(\Phi) \mathcal{W}^\alpha \mathcal{W}_\alpha) + \text{h.c.}\right].
\]

\section{Perturbative non-renormalization theorems}
We study the Wilsonian effective action
\[
  e^{-S_{eff}} = \int_{P > \lambda} \mathcal{D} A e^{i S(A)},
\]
where we integrate out all momenta above $\lambda$. The quantum corrections can be expanded perturbatively as
\[
  S_{eff} = S_{\text{tree level}} + \sum_n g^n S_{n\text{-loop}} + S_{\text{non-perturbative}}.
\]
Usually, people do not talk about the non-perturbative part, but it natural arises if you want to talk about instantons.

\begin{thm}
  Perturbatively,
  \begin{enumerate}
    \item $K$ gets corrected to each loop.
    \item $f(\Phi)$ gets corrections at $1$-loop, i.e.\ $f = f_{\mathrm{tree}} + f_{\mathrm{1-loop}}$.
    \item $W(\Phi)$ and $\xi$ do not get any corrections.
  \end{enumerate}
\end{thm}
The main ideas of the proof come from string theory.

\begin{proof}
  There are a few ideas:
  \begin{itemize}
    \item We introduce spurious superfields as a bookkeeping device
    \item Use all symmetries
    \item Use holomorphicity
    \item Take weak coupling limits.
  \end{itemize}
  We begin with spurious chiral superfields
  \[
    X = (x, \psi_x, F_x),\quad Y = (y, \psi_y, F_y).
  \]
  Consider the action
  \[
    S = \int \d^4 x \left[ \int \d^4 \theta (K + \xi) + \int \d^2 \theta (Y W(\Phi) + X \mathcal{W}^\alpha \mathcal{W}_\alpha) + \text{h.c.}\right].
  \]
  We write our effective Lagrangian as
  \[
    S_{\mathrm{eff}} = \int \d^4 x \left[\int \d^4 \theta \left[J(\Phi, \Phi^\dagger, e^V, X, Y) + \hat{\xi}(X, X^\dagger, Y, Y^\dagger) V\right] + \int \d^2 \theta [H(\Phi, X, Y \mathcal{W}^\alpha] + \text{h.c.}\right].
  \]
  Notice that $H$ is holomorphic.

  Recall that we had $R$-symmetry. This acts on our fields with charges
  \begin{center}
    \begin{tabular}{cccccccc}
      \toprule
      & $\Phi$ & $V$ & $X$ & $Y$ & $\theta$ & $\bar{\theta}$ & $\mathcal{W}^\alpha$\\
      \midrule
      charge & 0 & 0 & 0 & 2 & 1 & -1 & 1\\
      \bottomrule
    \end{tabular}
  \end{center}
  This does not commute with supersymmetry, because the $\theta$ transform.

  Since this symmetry has to be preserved in $S_{eff}$, we see that $H$ must be of the form
  \[
    H = Y h(X, \Phi) + g(X, \Phi) \mathcal{W}^\alpha \mathcal{W}_\alpha.
  \]
  The next symmetry is the Peccei--Quinn symmetry, given by
  \[
    X \mapsto X + i r,
  \]
  where $r$ is a real constant. Recall that
  \[
    X \mathcal{W}^\alpha \mathcal{W}_\alpha = \Re(X) F^{\mu\nu} F_{\mu\nu} + \Im X F^{^\mu\nu} \tilde{F}_{\mu\nu},
  \]
  and so $r F^{\mu\nu} \tilde{F}_{\mu\nu}$ is a total derivative. Since the system is invariant under a shift in $X$, and $h$ is holomorphic, we know that $h$ cannot be a function of $X$. We have to be a bit more careful for the second term, where we can allow a linear term in $X$ by the argument above. So we get
  \[
    H = Y h(\Phi) + (\alpha X + g(\Phi)) \mathcal{W}^\alpha \mathcal{W}_\alpha.
  \]
  This is key. Holomorphicity is what is forcing non-renormalization.

  Now in the limit $Y \to 0$, we must have $S \to S_{\mathrm{tree}}$. So we must have $h(\Phi) = W(\Phi)$. Similarly, taking $X \to 0$, we must have $\alpha = 1$. Now the number of powers of $X$ in each graph is
  \[
    N_X = V_X - I_X.
  \]
  Moreover, the number of loops is
  \[
    L = I_X - V_X + I = 1 - N_X.
  \]
  So we can only have $L = 0$ or $1$. So we are done.

  Finally, we have to talk about $\hat{\xi}$. We know this has to be constant to preserve supersymmetry and gauge invariance. But this can potentially be a constant other than the original $\xi$. To argue that this doesn't change, one argues that the correction is proportional to the sum of charges in the theory, which must vanish or else another diagram diverges. % fix this
\end{proof}

\section{Supersymmetry breaking}
We say supersymmetry is broken if the vacuum state is such that $Q_\alpha \bket{\Omega} \not= 0$. This is true iff $E > 0$.

\subsection{\texorpdfstring{$F$}{F}-term supersymmetry breaking}
If we have a chiral superfield $\Phi = \{\varphi, \psi, F\}$, then under supersymmetry, the transformations are given by
\begin{align*}
  \delta \varphi &= \sqrt{2} \varepsilon \psi\\
  \delta \psi &= \sqrt{2} \varepsilon F + i \sqrt{2} \sigma^\mu \bar{\varepsilon} \partial_\mu \varphi\\
  \delta F &= i \sqrt{2} \bar{\varepsilon} \bar{\sigma}^\mu \partial_\mu \psi.
\end{align*}
Note that to preserve Lorentz invariance, the vacuum expectation of any scalar field must be non-zero. So the only chance for non-zero vacuum energy is if $\bra F\ket \not= 0$. Then $\delta \psi \not= 0$. This $\psi$ is called the \term{Goldstone fermion}, or a Goldstino. This is consistent with our previous observation that the scalar potential is
\[
  V = FF^* = \left| \frac{\partial W}{\partial \psi} \right|^2.
\]
The simplest case where this happens is the \term{O'Raifeartaigh model}. This has three chiral superfields $\Phi_1, \Phi_2, \Phi_3$, with
\[
  K = \Phi_i^\dagger \Phi_i,\quad W = g \Phi_1( \Phi_3^2 - m^2) + M \Phi_2 \Phi_3,\quad M \gg m \not= 0.
\]
Then we have
\begin{align*}
  --F_1 &= \frac{\partial W}{\partial \varphi_1} = g (\varphi_3^2 - m^2)\\
  - F_2 &= \frac{\partial W}{\partial \varphi_2} = M \varphi_3\\
  - F_3 &= \frac{\partial W}{\partial \varphi_3} = 2g \varphi_1 \vavrphi_3 + M \varphi_2.
\end{align*}
We immediately see that it is impossible for $F_1 = F_2 = 0$, for we would simultaneously need $\varphi_3 = \pm m$ and $\varphi_3 = 0$. We can calculate
\[
  V = g^2 |\varphi_3^2 - m^2|^2 + M^2 |\varphi_3|^2 + |2g \varphi_1 \varphi_3 + M \varphi_2|^2.
\]
This has a minimum at
\[
  \bra \varphi_2\ket = \bra \varphi_3\ket = 0,
\]
and $\bra \varphi_1\ket$ can be arbitrary. So the minimum has a ``flat'' direction along $\varphi_1$. At the minimum, we have
\[
  \bra V\ket = g^2 m^4 > 0.
\]
We now want to compute the masses of the scalars, given by $\frac{\partial^2 V}{\partial \varphi_i \partial \varphi_j}$. We then find that
\[
  m_{\vaprhi_1} = 0,\quad m_{\varphi_2} = M,\quad \varphi_3 = a + ib,\quad m_a^2 = M^2 - 2g^2 m^2,\quad M_b^2 = M^2 + 2 g^2 m^2. % what does this mean?
\]
We see that there is a splitting of the masses. The masses of the fermions are given by
\[
  \left\bra \frac{\partial^2 W}{\partial \varphi_i \partial \varphi_j} \psi_i \psi_j =
  \begin{pmatrix}
    0 & 0 & 0\\
    0 & 0 & M\\
    0 & M & 0
  \end{pmatrix}.
\]
So we see that
\[
  m_{\psi_1} = 0,\quad m_{\psi_2} = m_{\psi_3} = M.
\]
So $\psi_1$ is the goldstino, which breaks the supersymmetry --- the mass of the fermion $\psi_3$ is no longer the same as the mass of the corresponding bosons $\varphi_3$. However, this cannot possibly model real life, because in this scenario, if $\psi_3$ is a quark, then the corresponding super-quark would be lighter than the actual quark, so we would have seen it already!

There are general results that this problem cannot be solved by renormalization group flow, i.e.\ the lighter boson cannot become heavier than the corresponding fermion particle. So we need to find other ways of breaking supersymmetry. % supersymmetry must be broken at tree level, because non-renormalization theorem says the potential doesn't change.

In general, the supertrace
\[
  \mathrm{Str} M^2 \equiv \sum (-1)^{2j + 1} (2j + 1) m_j^2 = 0,
\]
and this is not good if we want to break supersymmetry at tree level. This can be solved if we introduce supergravity.

There is also a $D$-term supersymmetry breaking for vector superfields $V = \{\lambda, A_\mu, D\}$ with $\delta \lambda = \varepsilon D$. Then $\bra D \ket \not= 0$ implies supersymmetry is broken, and $\lambda$ is the goldstino.

In supergravity, $V$ is not positive definite, and $\bra F \ket \not= 0$ and $\bra V \ket = 0$ is possible. The gravitino ``eats'' the godstino fermions. This is a super-Higgs effect (not to be confused with a supersymmetric Higgs effect, where the normal Higgs effect happens in a supersymmetric setting).

\section{The minimal supersymmetric standard model}
The minimal supersymmetric standard model is the simplest way to turn the standard model into a supersymmetric one. This has a super-gauge potential $V$ associated to $\SU_3 \times \SU_2 \times \SU_1$, and this has chiral superfields
\begin{align*}
  Q_i &= (3, 2, -1/6) & U_i^r % complete this from notes.
\end{align*}
We need two Higgs to cancel triangular anomalies. This has K\"ahler potential
\[
  K = \Phi^\dagger e^{2qV} \Phi,
\]
and potential
\begin{align*}
  W &= y_1 Q H_2 \tilde{u}^c + y_2 Q H_1 \tilde{d}^c + y_3 L H_1 \tilde{e}^c + \mu H_1 H_2 + W_{\slashed{BL}}\\
  W_{\slashed{BL}} &= \lambda_1 LL \tilde{e}^c + \lambda_@ L Q \tilde{d}^c + \lambda_3 \tilde{u}^e \tilde{d}^c \tilde{d}^c + \mu' LH_2.
\end{align*}
The $W_{\slashed{BL}}$ term breaks baryon and lepton number conservation. This is problematic, because it can lead to proton decay $p \to e^+ + \pi^0$. % inset diagram

To avoid this, we impose $R$-parity.

%\section{Physics in higher dimensions}
%\subsection{Bosons in higher dimensions}
%In higher dimensions, we have Poincar\'e generators $P^M, M^{NQ}$ where $M, N, Q = 0, \ldots, D - 1$. The algebra is formally the same, with
%\begin{align*}
%  [P^M, M^{NQ}] &= -i (\eta^{MN} P^Q - \eta^{MQ} P^N)\\
%  [M^{MN}, M^{PQ}] &= i(\eta M + \eta M - \eta M - \eta M)\\ % check
%\end{align*}
%Note in particular that $M^{2j, 2j + 1}$ commute with each other. They can then be individually diagonalized, and we have a generalized notion of spin, given by eigenvalues of $M^{23}, M^{45}, \ldots$. If $\mathcal{O}$ is an operator with
%\[
%  [\mathcal{O}, M^{01}] = i \omega \mathcal{O},
%\]
%we say the \term{weight} of $\mathcal{O}$ is $\omega$.
%
%The massless particles can be chosen to have momentum
%\[
%  P^M = (E, E, 0, \ldots, 0).
%\]
%The little subgroup is then an extension of $\SO(D - 2)$. We can have the following types of bosonic fields:
%\begin{enumerate}
%  \item Real scalar fields $\varphi(x^n)$ with helicity $0$ and $1$ degree of freedom.
%  \item Vector fields $A_M (x^M)$. This has helicities $\lambda = 1, 0$ (depending on the direction we are looking at). For example, for $J_{23}$, the eigenvalue of $A_i$ is $\lambda = 1 $ whenever $i = 2, 3$, and $0$ for $m = 4, \ldots, D - 1$.
%
%    The number of degrees of freedom is $D - 2$.
%  \item The graviton $g_{MN}(x^M)$ has helicities $2, 1, 0$. The number of degrees of freedom is $ \frac{(D - 2)(D - 1)}{2} - 1 = \frac{D(D - 3)}{2}$.
%  \item In general, we can have $p$-forms for all $p$. For example, in $D$ dimensions, for a $2$-form, $A_{ij}$ and $A_{mn}$ have $\lambda = 0$, while $A_{im}$ has $\lambda = 1$. % i = 2, 3; m = 4, \ldots, D - 1
%  The number of degrees of freedom is $\binom{D - 2}{p}$.
%
%\end{enumerate}
\printindex
\end{document}
